\documentclass[10pt]{beamer}

\usetheme[progressbar=frametitle]{metropolis}

\metroset{%
  block=fill,%
  background=dark,%
  titleformat title=smallcaps,%
  titleformat subtitle=smallcaps,%
  titleformat section=smallcaps,%
  titleformat frame=smallcaps%
}

\usepackage{booktabs}
\usepackage{hyperref}
\usepackage{graphicx}
\usepackage{tikzsymbols}
\usepackage[numbers]{natbib}
\usepackage[francais]{babel}

\DeclareGraphicsExtensions{.pdf,.png,.jpg,.jpeg}
\graphicspath{{./img/}}

\hypersetup{%
  colorlinks=true,%
  linkcolor=white,%
  urlcolor=cyan%
}

\newcommand{\imgtw}[2][1]{%
  \begin{figure}%
    \centering%
    \includegraphics[width=#1\textwidth]{#2}%
  \end{figure}%
}
\newcommand{\imgth}[2][1]{%
  \begin{figure}%
    \centering%
    \includegraphics[height=#1\textheight]{#2}%
  \end{figure}%
}
\newcommand{\imgcmw}[2]{
  \begin{figure}%
    \centering%
    \includegraphics[width=#1cm]{#2}%
  \end{figure}%
}
\newcommand{\imgcmh}[2]{
  \begin{figure}%
    \centering%
    \includegraphics[height=#1cm]{#2}%
  \end{figure}%
}
\newcommand{\img}[1]{
  \begin{figure}%
    \centering%
    \includegraphics{#1}%
  \end{figure}%
}

\title{Ivre, il pratiquait le ML sans recul}
\subtitle{}
\date{4 février 2019}
\author{\textsc{Hugo Mougard}}
\institute{\textsc{Nantes Machine Learning Meetup}}

\begin{document}

\maketitle

\begin{frame}[standout]
  \frametitle{Avant-propos}
  Je suis

  \begin{itemize}
  \item ingé R\&D ML à \href{https://sourced.tech}{source\{d\}}
  \item co-organisateur de ce meetup
  \end{itemize}
\end{frame}

\begin{frame}[standout]
  \frametitle{Avant-propos}
  Je \emph{ne} suis \emph{pas}

  \begin{itemize}
  \item climatologue
  \item sociologue
  \item magistrat
  \item …
  \end{itemize}
\end{frame}

\begin{frame}[standout]
  \frametitle{Avant-propos}
  \onslide<+->{Je reviens de la FOSDEM}\onslide<+->{, en avion

  \vspace{1cm}

  \Walley[4]}
\end{frame}

\begin{frame}{Plan de la présentation}
  \tableofcontents
\end{frame}

\section{Introduction}
\label{sec:introduction}

\begin{frame}{But}
  Examiner le ML sous quelques angles critiques
\end{frame}

\begin{frame}{Quelques points de départ}
  \begin{itemize}
  \item Collapsologie
  \item Rapports du GIEC sur le Climat
  \item Révélations d'\textsc{Edward Snowden}
  \item Climat social
  \item Épisode de \textsc{Data Gueule} \emph{Algocratie}
  \end{itemize}
\end{frame}

\begin{frame}
  \frametitle{Collapsologie}

  Étude de l'effondrement de la société thermo-industrielle.

  \href{https://www.meetup.com/Collapsologie-Nantes/}{Meetup} sur
  Nantes.
\end{frame}


\begin{frame}
  \frametitle{Rapport du GIEC}

  \href{https://www.ipcc.ch/sr15/}{Rapport} spécial sur le
  réchauffement d'1,5\degres C (2018).
\end{frame}

\begin{frame}
  \frametitle{Révélations d'Edward Snowden}
  \imgth[.7]{edward}
  Omnipotence de la surveillance (2014).
\end{frame}

\begin{frame}
  \frametitle{Climat social}

  Gilets jaunes, optimisation fiscale, augmentation des inégalités, …
\end{frame}

\begin{frame}
  \frametitle{Algocratie}

  \imgtw[.5]{algocratie}

  Épisode 84, \href{https://youtu.be/oJHfUv9RIY0?t=524}{\emph{Algocratie}}
\end{frame}

\section{Le ML est-il important ?}
\label{sec:le-ml-est-il-important}

\begin{frame}
  \frametitle{Question}
  Pour X, Y et Z,

  \vspace{1cm}

  \centering
  \emph{le ML est-il \alert{vraiment} important ?}
\end{frame}

\begin{frame}
  \frametitle{Changeons de perspective historique}
  Replaçons les algorithmes au centre\footnote{Basé \emph{très
      vaguement} sur une
    \href{https://docs.google.com/presentation/d/1XqMVwzmQpQL7gxz-75GYchZ\_BdzBmk2r3-C3NDTHYvQ/edit\#slide=id.g41eb28881a\_0\_16}{présentation}
    de Thierry Caminel.}.
\end{frame}

\begin{frame}
  \frametitle{Évolution génétique}
  \imgth[.5]{ape}
  Accumulation biochimique des algorithmes (\emph{très} lente).
\end{frame}

\begin{frame}
  \frametitle{Histoire}
  \imgtw[.7]{tool-1}
  Accumulation culturelle des algorithmes (lente).
\end{frame}

\begin{frame}
  \frametitle{Histoire moderne}
  \imgtw[.7]{tool-2}
  Accumulation explicite des algorithmes ---~informatique (rapide).
\end{frame}

\begin{frame}
  \frametitle{Maintenant}
  \imgtw[.7]{alphago}
  Accumulation semi-automatique des algorithmes ---~AI/ML (très rapide).
\end{frame}

\begin{frame}
  \frametitle{Peut-être un jour}
  \imgtw[.9]{hal}
  Accumulation automatique des algorithmes ---~AGI (quasi instantanée).
\end{frame}

\begin{frame}
  \frametitle{Changeons de perspective économique}
  Capital dominant : algorithmes.
\end{frame}

\begin{frame}
  \frametitle{Meilleures valorisations en bourse}
  \centering
  \begin{tabular}{rl}
    \toprule
    2T 2007          & 2T 2018          \\
    \midrule
    PetroChina       & \emph{Apple}     \\
    Exxon            & \emph{Amazon}    \\
    GE               & \emph{Google}    \\
    China~Mobile     & \emph{Microsoft} \\
    Bank~of~China    & Berkshire        \\
    \emph{Microsoft} & \emph{Facebook}  \\
    Gazprom          & \emph{Alibaba}   \\
    Shell            & \emph{Tenscent}  \\
    AT\&T            & JP~Morgan        \\
    \bottomrule
  \end{tabular}
\end{frame}

\begin{frame}
  \frametitle{Importance du ML}
  \begin{align*}
    \text{Algorithmes} & = \text{enjeu majeur}      \\
    \text{ML}          & = \text{enjeu majeur} \\
  \end{align*}
\end{frame}

\section{Coût carbone}
\label{sec:cout-carbone}

\begin{frame}
  \frametitle{Impact du ML}
  Utilisation de l'impact du numérique comme proxy\footnote{Toute
    cette section est basée sur le \href{https://theshiftproject.org%
      /wp-content/uploads/2018/10%
      /2018-10-04_Rapport_Pour-une-sobri\%C3\%A9t\%C3\%A9%
      -num\%C3\%A9rique_Rapport_The-Shift-Project.pdf}%
    {Rapport pour une sobritété numérique} de \textsc{The Shift
      Project}}.
\end{frame}

\begin{frame}
  \frametitle{Background: plan}
  \imgtw{plan}
\end{frame}

\begin{frame}
  \frametitle{Background: part du numérique}
  4\% (2\% pour l'aéronautique domestique, 8 pour le parc automobile).
\end{frame}

\begin{frame}
  \frametitle{Effet \textsc{YouTube}}

  Impact énergétique de la vue d'une vidéo: \textbf{1500} fois plus grand que la
  consommation du smartphone
\end{frame}

\begin{frame} 
  \frametitle{Effet \textsc{YouTube} ---~\textsc{Google Translate} vu
    par un utilisateur}

  \imgth[.9]{translate-mobile}
\end{frame}

\begin{frame}
  \frametitle{Effet \textsc{YouTube} ---~\textsc{Google Translate} vu
    par un data scientist}

  \imgtw{datascientist-view}
\end{frame}

\begin{frame}
  \frametitle{Effet rebond}

  Accroître le rendement énergétique d'un objet \textbf{augmente} la
  consommation d’énergie globale dédiée à la fonction technique que
  remplit cet objet.
\end{frame}

\begin{frame}
  \frametitle{Coût de production}

  Pour un smartphone, 33 fois supérieur à sa consommation électrique
  annuelle.

\end{frame}

\begin{frame}
  \frametitle{Consommation du numérique en croissance}
  \imgtw{consumption}
\end{frame}

\begin{frame}
  \frametitle{Part du numérique en croissance}
  \imgtw[.7]{share}
\end{frame}


\section{Technocratie}
\label{sec:technocratie}

\begin{frame}
  \frametitle{Mode de gouvernance}
  \begin{itemize}[<+->]
  \item par les experts
  \item méritocratie
  \item productivité
  \end{itemize}
\end{frame}

\begin{frame}
  \frametitle{Comparatif réducteur, subjectif \& superficiel}

  \begin{figure}
    \centering
    \begin{tabular}{lll}
      \toprule
                         & Technocracie                   & Démocratie               \\
      \midrule
      Question optimisée & Comment                        & Quoi                     \\
      Décideurs          & Experts                        & Citoyens / représentants \\
      Valeur optimisée   & Intérêt sectoriel              & Intérêt général          \\
      \bottomrule
    \end{tabular}
  \end{figure}
\end{frame}

\begin{frame}
  \frametitle{Exemple}
  \textsc{NaonedIA}
\end{frame}

\begin{frame}
  \frametitle{Rôle du ML en technocratie}

  \imgtw[.7]{sky}

  ML = expert parfait $\rightarrow$ renforcement de la technocratie.
\end{frame}

\begin{frame}
  \frametitle{À éviter}
  \imgtw[.9]{simpson}
\end{frame}

\section{Asymétrie des acteurs}
\label{sec:asymetrie-des-acteurs}

\begin{frame}
  \frametitle{Question}
  \imgth[.7]{david-goliath}
  Tous les acteurs peuvent-ils utiliser le ML aussi efficacement ?
\end{frame}

\begin{frame}
  \frametitle{Citoyens vs états: sûreté et sécurité}

  Trade-off entre libertés individuelles et contrôle
  étatique\footnote{Basé sur l'excellent billet de blog de
    Maître~Eolas
    \href{http://www.maitre-eolas.fr/post/2015/04/06/Relisons-la-notice}{\emph{Relisons
        la notice}}.}.

  \begin{description}
  \item[Sûreté] Protection contre l'état
  \item[Sécurité] Protection contre les autres citoyens
  \end{description}

\end{frame}

\begin{frame}
  \frametitle{Citoyens vs états: sûreté et sécurité}

  \onslide<+->{Quelques stats sur la NSA}

  \begin{itemize}[<+->]
  \item $\approx$ 11 milliards en 2013 (Snowden)
  \item Datacenter en Utah: des Exa-octets en stockage ($10^{18}$
    octets)
  \item 30k à 40k employés
  \end{itemize}

  \onslide<+->{$\rightarrow$ Fort déséquilibre en faveur de la
    sécurité.}
\end{frame}

\begin{frame}
  \frametitle{Citoyens vs états: sûreté et sécurité}

  \onslide<+->{Et en France}

  \begin{itemize}[<+->]
  \item Installation de boîtes noires chez les FAI
  \item Renforcement du pouvoir administratif
  \end{itemize}

  \onslide<+->{$\rightarrow$ En sûreté, recours au ML compliqué}
\end{frame}

\begin{frame}
  \frametitle{TPE vs multi-nationales}

  \onslide<+->{Coût d'entrée élevé en ML dû à}

  \begin{itemize}[<+->]
  \item la force de calcul nécessaire (\textsc{AlphaGo})
  \item la quantité de données nécessaires (\textsc{Criteo})
  \item la quantité d'utilisateurs nécessaire (\textsc{Waze})
  \end{itemize}

  \onslide<+->{$\rightarrow$ Effet monopole accentué.}
\end{frame}

\section{Biais et ML}
\label{sec:biais-et-ml}

\begin{frame}
  \frametitle{Un autre biais}

  \onslide<+->{Dû à l'entraînement depuis les données, risque de
    biais.}

  \begin{itemize}[<+->]
  \item pas biais \emph{statistique}
  \item biais \emph{social}
  \item possibilité d'être biaisé statistiquement et pas socialement, ou l'inverse.
  \end{itemize}
\end{frame}

\begin{frame}
  \frametitle{Reproduction de biais}

  \onslide<+->{Recette classique:}

  \begin{enumerate}[<+->]
  \item données biaisées $\rightarrow$ modèle biaisé
  \item modèle biaisé $\rightarrow$ nouvelles données biaisées
  \item \texttt{goto} 1., potentiellement avec intérêts
  \item \emph{Profit}. Biais éternel
  \end{enumerate}
\end{frame}

\begin{frame}
  \frametitle{Création de biais social}

  \onslide<+->{Cas \textsc{YouTube}}

  \begin{itemize}[<+->]
  \item optimise le temps passé sur le site
  \item vidéo polémique $>$ vidéo normale pour ce critère
  \item biais social énorme vers les complot, Trump, etc, …
  \end{itemize}

  \onslide<+->{$\rightarrow$ Facteur important de l'élection de Trump.}
\end{frame}

\section{Pistes d'action}
\label{sec:piste-d-action}

\begin{frame}
  \frametitle{Transition numérique sobre}
  Directement du \href{https://theshiftproject.org%
    /wp-content/uploads/2018/10%
    /2018-10-04_Rapport_Pour-une-sobri\%C3\%A9t\%C3\%A9%
    -num\%C3\%A9rique_Rapport_The-Shift-Project.pdf}{rapport \emph{Pour une
    sobriété numérique}}:
  \begin{itemize}[<+->]
  \item adopter la sobriété numérique comme principe d'action
  \item accélérer la prise de conscience des impacts environnementaux
    du numérique
  \item permettre aux organisations de piloter environnementalement
    leur transition numérique
  \item procéder à un bilan carbone des projets numériques
  \item améliorer la prise en compte des aspects systémiques du
    numérique dans les secteurs clefs
  \item mettre en place des mesures à l'échelle européenne
  \end{itemize}
\end{frame}

\begin{frame}{Prendre conscience du trade-off features/coût
    écologique}
  \begin{itemize}[<+->]
  \item intégrer le coût équivalent carbone comme métrique
    décisionnelle
  \item régulièrement, à plusieurs échelles
  \item en prenant en compte les effets \textsc{YouTube}, rebond, etc
  \end{itemize}
\end{frame}

\begin{frame}{Prendre conscience du trade-off features/coût
    écologique}

  \imgth[.9]{do-or-not}
\end{frame}

\begin{frame}{Serment d'Hippocrate du Data Scientist}
   \url{https://hippocrate.tech/}

   \pause

   \begin{enumerate}[<+->]
   \item Intégrité scientifique et rigueur
   \item Transparence
   \item Équité
   \item Respect
   \item Responsabilité et indépendance
   \end{enumerate}
\end{frame}

\begin{frame}{Droit de retrait}
  Lié à l'indépendance du Serment d'Hippocrate du Data Scientist.
\end{frame}

\begin{frame}{Lutter contre les biais}
  Consultation des citoyens, sociologues, magistrats, …
\end{frame}

\begin{frame}{S'informer, communiquer}
  Très peu d'infos disponibles encore sur le coût carbone et social du
  ML.
\end{frame}

\begin{frame}{Transparence}
  Communiquer à vos utilisateurs l'impact de votre utilisation du ML.
\end{frame}

\begin{frame}[standout]
  Merci pour votre attention !
\end{frame}

\begin{frame}[standout]
  Questions / débat time !
\end{frame}



\end{document}

%%% Local Variables:
%%% mode: latex
%%% TeX-master: t
%%% End:
